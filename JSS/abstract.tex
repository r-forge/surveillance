\Abstract{
The availability of geocoded health data and the inherent temporal structure of
communicable diseases have led to the development of a whole range of statistical
models for the spatio-temporal analysis of infectious diseases. Aim is to describe and
analyse such data, typically collected as part of routine public health surveillance systems,
in order to better understand disease dynamics and the relationship with risk factors and
interventions. The present work focuses on functionality of the open source \proglang{R} package
\pkg{surveillance} available from CRAN to accomplish this task.

We distinguish data by its available resolution in space and time. In its most natural form,
event data are available continuous in space within the selected observation region and
continuous in time. However, in some applications, the set of locations where events can
occur is discrete, e.g., for epidemics among households or farm animals. The \code{S3} classes
\code{"epidataCS"} (continuous space) and \code{"epidata"} (discrete space) represent such point
process event data, possibly with the specification of additional external covariates which
may vary in space and/or time. Variations of Hawkes self-exciting point-process consisting
of endemic and epidemic components parametrized by covariates are used for spatio-temporal epidemic modelling.
Maximum likelihood estimation for these models is
implemented in functions \code{twinstim} and \code{twinSIR}. Finally, if data are available in
temporal and geographical aggregated form, i.e., as a multivariate time series of counts,
the \code{S4} class \code{"sts"} can be used to represent data and the function \code{hhh4} allows for similar
endemic-epidemic time-series modelling as above. In the latter case, spatial correlation in
the time series can also be covered by additional random effects in a conditional
autoregressive formulation.

The implemented methods are illustrated using surveillance data on invasive
meningococcal disease in Germany 2002--2008, the Hagelloch 1861 measles outbreak,
and influenza in southern Germany 2001--2008. Altogether, the \pkg{surveillance} package
contains a comprehensive set of tools for spatio-temporal analyses of epidemic phenomena.
}

\Keywords{space-time modelling, infectious diseases, epidemics, point process, multivariate time series of counts}
%\Plainkeywords{keywords, comma-separated, not capitalized, Java} %% without formatting
%% at least one keyword must be supplied
